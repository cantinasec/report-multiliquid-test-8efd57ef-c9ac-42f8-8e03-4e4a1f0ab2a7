\hypertarget{overview}{%
\section{Overview}\label{overview}}

\subsection{Overall Security Posture \&
Maturity}\label{overall-security-posture--maturity}

The security review of the Multiliquid Solana swap program indicates a
moderate level of security maturity. While no critical or high-risk
vulnerabilities were identified, several areas for improvement were
noted across different risk categories. The project team demonstrated
responsiveness by promptly addressing the most significant medium-risk
finding and the majority of low-risk and informational issues identified
during or shortly after the review period. This proactive approach
reflects a commitment to enhancing the protocol\textquotesingle s
security.

\subsection{Major Findings and Risks}\label{major-findings-and-risks}

\begin{itemize}
\tightlist
\item
  \textbf{U64DynamicAddress Pricing cannot distinguish between assets}
  This issue meant the system could incorrectly determine the value of
  assets, potentially causing users to swap at inaccurate prices. This
  could lead to financial losses for users and erode trust in the
  platform\textquotesingle s fairness and reliability.
\item
  \textbf{Global config initialization can be frontrun} A malicious
  actor could potentially initiate the system\textquotesingle s global
  configuration with their own parameters before the legitimate
  administrators. If undetected, this could lead to a loss of
  operational control, financial misdirection, such as diverting fees,
  and significant reputational damage to the Multiliquid platform.
\end{itemize}

\subsection{General Business Impact}\label{general-business-impact}

The identified issues primarily posed risks to the financial integrity
and operational stability of the Multiliquid swap program. Incorrect
asset pricing, if unaddressed, could lead to financial losses for users,
directly impacting customer trust and potentially drawing negative
public attention. The risk of a malicious actor frontrunning the
system\textquotesingle s initialization could lead to unauthorized
control over critical parameters, resulting in financial theft or
operational disruption, with severe financial and reputational
consequences for Multiliquid. While many issues have been resolved,
ongoing vigilance and careful deployment procedures are crucial to
maintain a robust and trustworthy platform.

\subsection{Review Outcome}\label{review-outcome}

The security review concluded with most identified findings, including
the highest-severity medium risk and several low and informational
items, being successfully addressed by the development team. The
significant risk related to global configuration initialization being
frontrun was acknowledged. This indicates the team is aware of this
potential vulnerability and must implement robust deployment safeguards
and validation procedures to mitigate this risk effectively, ensuring a
secure launch and ongoing operation.

