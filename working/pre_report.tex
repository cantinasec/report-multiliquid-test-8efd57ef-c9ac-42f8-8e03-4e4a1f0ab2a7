\hypertarget{introduction}{%
\section{Introduction}\label{introduction}}


\hypertarget{about_cantina}{
\subsection{About Cantina}\label{about cantina}}
Cantina is a security services marketplace that connects top security
researchers and solutions with clients. Learn more at
\href{https://cantina.xyz}{cantina.xyz}


\hypertarget{disclaimer}{
\subsection{Disclaimer}\label{disclaimer}}
Cantina Managed provides a detailed evaluation of the security posture
of the code at a particular moment based on the information available at
the time of the review. While Cantina Managed endeavors to identify and
disclose all potential security issues, it cannot guarantee that every
vulnerability will be detected or that the code will be entirely secure
against all possible attacks. The assessment is conducted based on the
specific commit and version of the code provided. Any subsequent
modifications to the code may introduce new vulnerabilities that were
absent during the initial review. Therefore, any changes made to the
code require a new security review to ensure that the code remains
secure. Please be advised that the Cantina Managed security review is
not a replacement for continuous security measures such as penetration
testing, vulnerability scanning, and regular code reviews.
% Unused: Multiliquid   solana   swap   program



% SUBSECTION: Risk Assessment
\hypertarget{risk-assessment}{%
\subsection{Risk assessment}\label{risk-assessment}}


% Width of table lines to 0.5mm
\setlength{\arrayrulewidth}{0.3mm}
% Padding inside cell
\renewcommand{\arraystretch}{1.1}

% --> Severity Classiification table: <--
\begin{center}

\begin{tabular}{|l|l|l|l|}
  \hline \textbf{Severity level} & \textbf{Impact: High} & \textbf{Impact: Medium} & \textbf{Impact: Low} \\
  \hline \textbf{Likelihood: high} & Critical & High & Medium \\
  \hline \textbf{Likelihood: medium} & High & Medium & Low \\
  \hline \textbf{Likelihood: low} & Medium & Low & Low \\
  \hline
\end{tabular}
\end{center}


% SUB-SUB-SECTIONS
\hypertarget{severity}{%
\subsubsection{Severity Classification}\label{severity}}

The severity of security issues found during the security review is 
categorized based on the above table. Critical findings have a high 
likelihood of being exploited and must be addressed immediately. 
High findings are almost certain to occur, easy to perform, 
or not easy but highly incentivized thus must be fixed as soon as possible.

Medium findings are conditionally possible or incentivized but are still relatively likely
to occur and should be addressed. Low findings are a rare combination of circumstances to exploit, 
or offer little to no incentive to exploit but are recommended to be addressed.

Lastly, some findings might represent objective improvements that should be addressed 
but do not impact the project's overall security (Gas and Informational findings).

